\documentclass[12pt,letterpaper]{article}
\usepackage[a4paper, total={6.5in, 10in}]{geometry}
\usepackage{graphicx}
\usepackage[export]{adjustbox}

\graphicspath{{./picture/}}
\usepackage{xeCJK}
\usepackage{float}
\usepackage[colorlinks=true,linkcolor=blue,urlcolor=cyan]{hyperref}

\title{报告二}
\author{王瑞洁 24020007127}
\date{2025.09.15}

\begin{document}
\maketitle
\section{实验内容}
\subsection{python入门基础}
\subsection{python视觉运用}
\subsection{命令行环境}
\vspace{1em}  %换行

\section{实验实例(20)}

\subsection{实例一}
题目:有四个数字1、2、3、4,能组成多少个互不相同且无重复数字的三位数?各是多少?
\begin{figure}[H]
\centering
\includegraphics[width=10cm,height=2.5cm]{01}
\end{figure}
将个十百位数放入1、2、3、4组成的三位数字所有情况排出来,再根据无重复数字的要求筛选满足条件的三位数字。运行结果展示如下:
\begin{figure}[H]
\centering
\includegraphics[width=10cm,height=8cm]{02}
\end{figure}

\subsection{实例二}
题目:计算奖金利润提成。利润低于10万,提成10\%;高于10万,低于20万,该部分提成7.5\%;高于20万,低于30万,该部分提成5\%;高于30万,该部分提成1\%。根据输入的利润求发放的奖金额。
\begin{figure}[H]
\centering
\includegraphics[width=10cm,height=7cm]{03}
\end{figure}
arr存放奖金计算临界点,对应奖金区间的利润比例存放在rat里,有四个临界点,循环四次,比较输入的利润值和计算临界点,累加各部分的奖金分成,并打印出来,循环结束后显示最终的奖金总额。

\subsection{实例三}
题目:计算一个整数,加上100后是一个完全平方数,再加上168后也是一个完全平方数。
\begin{figure}[H]
\centering
\includegraphics[width=10cm,height=8cm]{04}
\end{figure}
思路如下:
\begin{figure}[H]
\centering
\includegraphics[width=10cm,height=4cm]{05}
\end{figure}

\subsection{实例四}
题目:输入某年某月某日,判断这一天是这一年的第几天
\begin{figure}[H]
\centering
\includegraphics[width=10cm,height=9cm]{06}
\end{figure}

\subsection{实例五}
题目:排序。输入三个整数,按由小到大的顺序排列。
\begin{figure}[H]
\centering
\includegraphics[width=10cm,height=7cm]{07}
\end{figure}
使用append函数向列表numbers末尾添加元素,使用sort函数对添加完元素的列表排序。

\subsection{实例六}
题目:将一个数组逆序输出
\begin{figure}[H]
\centering
\includegraphics[width=10cm,height=8cm]{08}
\end{figure}
将首位的数字依次对换位置,a[j],a[length-j-1]=a[length-j-1],a[j],,只用循环数组长度的“一半(数组长度除以2取整)”次数即可。

\subsection{实例七}
题目:判断101到150之间有多少个素数,并输出全部素数
\begin{figure}[H]
\centering
\includegraphics[width=10cm,height=9cm]{09}
\end{figure}

\subsection{实例八}
题目:根据输入的额分数,判定学生成绩的等级,>=90属于A,60~89属于B,<=60属于C。要求使用条件运算符的嵌套。
\begin{figure}[H]
\centering
\includegraphics[width=10cm,height=7.5cm]{10}
\end{figure}

\subsection{实例九}
题目:求一个3*3矩阵主对角线元素之和
\begin{figure}[H]
\centering
\includegraphics[width=10cm,height=8.5cm]{11}
\end{figure}

\subsection{实例十}
题目:数组插入。给定一个已经排好有小到大的数组,输入一个数字,按由小到大的顺序插入数组,并输出
\begin{figure}[H]
\centering
\includegraphics[width=10cm,height=7cm]{12}
\end{figure}

\subsection{实例十一}
在cmd里输入pip install pillow安装pillow库\\
我这里用的是pycharm,在终端中输入pip install pillow安装pillow库
\begin{figure}[H]
\centering
\includegraphics[width=10cm,height=7cm]{13}
\end{figure}
利用pillow库更改图片的对比度
\begin{figure}[H]
\centering
\includegraphics[width=10cm,height=6.5cm]{14}
\end{figure}
\vspace{-6mm}
\begin{figure}[H]
\centering
\includegraphics[width=10cm,height=8cm]{15}
\end{figure}

\subsection{实例十二}
同样的方法安装numpy
\begin{figure}[H]
\centering
\includegraphics[width=10cm,height=2cm]{16}
\end{figure}
实现RGB通道分离和展示效果
\begin{figure}[H]
\centering
\includegraphics[width=10cm,height=6.5cm]{17}
\end{figure}
\vspace{-6mm}
\begin{figure}[H]
\centering
\includegraphics[width=10cm,height=4.5cm]{18}
\end{figure}

\subsection{实例十三}
同样的方法安装scipy库和matplotlib库
\begin{figure}[H]
\centering
\includegraphics[width=10cm,height=2cm]{19}
\end{figure}
先灰度处理以便简化模糊
\begin{figure}[H]
\centering
\includegraphics[width=10cm,height=10cm]{20}
\end{figure}
\vspace{-6mm}
\begin{figure}[H]
\centering
\includegraphics[width=10cm,height=5.5cm]{21}
\end{figure}

\subsection{实例十四}
同样的方法安装opencv-python库
\begin{figure}[H]
\centering
\includegraphics[width=10cm,height=4.5cm]{22}
\end{figure}
利用opencv-python库为实现图片裁剪以及裁剪后的比例调整
\begin{figure}[H]
\centering
\includegraphics[width=10cm,height=6cm]{23}
\end{figure}
\vspace{-6mm}
\begin{figure}[H]
\centering
\includegraphics[width=10cm,height=6cm]{24}
\end{figure}

\subsection{实例十五}
先安装ImageMagick,再用和前面实例同样的方法安装wand库
\begin{figure}[H]
\centering
\includegraphics[width=10cm,height=2cm]{25}
\end{figure}
利用wand库实现图像的优化效果和浮雕效果
\begin{figure}[H]
\centering
\includegraphics[width=10cm,height=6.5cm]{26}
\end{figure}
\vspace{-6mm}
\begin{figure}[H]
\centering
\includegraphics[width=10cm,height=6cm]{27}
\end{figure}

\subsection{实例十六}
题目要求:(1)执行sleep 10000,用Ctrl+z暂停并切换到后台\\
(2)用pgrep找进程 PID,用pkill结束进程\\
(3)编写pidwait函数:接受PID为参数,等待进程结束
\begin{figure}[H]
\centering
\includegraphics[width=10cm,height=4cm]{28}
\end{figure}
在zshrc脚本末尾添加这段代码:
\begin{figure}[H]
\centering
\includegraphics[width=10cm,height=5cm]{29}
\end{figure}
先启动sleep 60 \&这个后台进程,根据输出的样例,调用pidwait等待这个PID。
\begin{figure}[H]
\centering
\includegraphics[width=10cm,height=2.5cm]{30}
\end{figure}

\subsection{实例十七}
题目要求:\\
(1)完成tmux教程,自定义tmux配置\\
(2)熟练使用tmux分屏、切换窗口、分离/链接会话\\
先新建一个会话并分屏。按ctrl+b组合键,在立即按v,会垂直分屏,按ctrl+b+右方向键会切换到右侧面板,在右侧面板输入ls,再按ctrl+b+左方向键会回到左侧面板。
\begin{figure}[H]
\centering
\includegraphics[width=10cm,height=5.5cm]{31}
\end{figure}

\subsection{实例十八}
题目要求:\\
(1)创建dc别名,实现输入dc等同于cd\\
(2)查看最常用的8条命令,为他们创建别名\\
在zshrc脚本里面添加如下部分,最后生效配置即可实现
\begin{figure}[H]
\centering
\includegraphics[width=10cm,height=4cm]{32}
\end{figure}

\subsection{实例十九}
题目要求:配置文件dotfiles管理\\
(1)新建Dotfiles文件夹,用git版本控制\\
(2)添加至少一个配置文件,包含自定义设置\\
(3)编写安装脚本,实现新设备快速部署\\
先初始化dotfiles仓库,新建projects文件夹并初始化git,移动配置文件到仓库,再创建软连接
\begin{figure}[H]
\centering
\includegraphics[width=10cm,height=10cm]{33}
\end{figure}
编写install.sh脚本,保存退出,chmod +x添加可执行权限
\begin{figure}[H]
\centering
\includegraphics[width=10cm,height=7cm]{34}
\end{figure}

\subsection{实例二十}
配置ssh环境,ssh免密登录配置(主机到虚拟机)\\
(1)在windows主机生成SSH密钥对,生成后主机~/.ssh目录下会有一个私钥文件和一个公钥文件,私钥文件不可泄露,公钥文件要传到虚拟机上
\begin{figure}[H]
\centering
\includegraphics[width=10cm,height=7cm]{35}
\end{figure}
(2)将公钥传到虚拟机上,使用ssh-copy-id工具,将公钥添加到虚拟机的授权列表
\begin{figure}[H]
\centering
\includegraphics[width=10cm,height=3.5cm]{36}
\end{figure}
(3)配置主机的别名,编辑主机的~/.ssh/config文件,添加到虚拟机的链接配置,之后只用输入ssh vm就可以访问

\section{总结和体会}
本次实验围绕python基础编程、python视觉运用以及命令行环境三大核心主题,在python基础部分,我熟练掌握了python语言环境下循环嵌套、条件判断、列表等操作,学习使用了算法优化来降低程序的复杂度;在python视觉部分,我学习使用了pillow库、numpy库、opencv-python库和wand库,分别完成了图像对比度调整、RGB通道分离、图像裁剪、浮雕和油画效果显示,体会到了多库协同在视觉任务中的优势;在命令行环境部分,我遇到了很多问题,例如在配置环境时,遇到了git终端中github身份验证的问题,查询后发现,git操作不再支持密码验证,于是我通过生成github个人访问令牌,并在push时用令牌代替密码解决了这个问题.这次实验不仅使我掌握了具体的技术和工具使用,也培养了我的系统思维和解决问题的能力。

\end{document}